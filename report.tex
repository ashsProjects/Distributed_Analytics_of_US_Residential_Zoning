\documentclass[titlepage]{article}
\usepackage[a4paper, total={6in, 9in}]{geometry}
\setlength{\tabcolsep}{11pt}
\usepackage{float}

\begin{document}

\title{
  CS455: Term Project Report\\
  \large Analysis of Residential Zoning on\\
    Quality of Life and Standard of Living}

\author{Ayush Adhikari, Brendan Verspohl}

\maketitle

\section{Introduction}
Zoning has been used for many decades in the United States to separate industrial, commercial, and residential sectors, but it was also once a motivation to segregate people of different races and classes. Governments designated some areas for residential use, which were predominantly used to build spacious single-family homes that appealed to middle-class families. This suburban development, which often prohibited multi-family homes and imposed a minimum lot size, was appealing to white families wanting to escape the urban environment that was increasing in racial diversity and limited housing options. Families with lower income could not afford to live in these areas and were pushed towards redlined areas. Through the expansion of highways, suburbs became more accessible and popular, but it also hindered progress towards inclusive communities as suburbs became isolated and reliant on cars rather than public transport \cite{freemark}. Most developed European and Asian cities have a longer history than the United States and employ a more inclusive zoning where buildings are not strictly used for one purpose. This allows residents to be closer to commerce and one another, often reducing the need for cars and long commute times, and therefore, the carbon footprint \cite{romem}. 

With an effective zoning policy, governments can incentivize subsidized housing, public transportation, better land use, and policies that discourage displacement of people with low incomes. With this project, we look to compare the effects on quality of life and standard of living as a result of single-family housing in different areas of the United States, and possibly the world. We considered aspects such as unemployment, commute, poverty, mental health, occupancy, lot sizes, rent, etc. While some of these are harder to measure and compare, others provided a good insight into our hypothesis that single family residential zoning results in a worse standard of living \cite{uscensus}. For the analysis, we employed Spark with Java and Gradle and Python for the final presentation and discussion of our results. 

This paper is structured  as follows. Section 2 defines the technical description of the problem and section 3 identifies approaches and implementations with their advantages and disadvantages of that problem. We detail more on our implementation and the tools we used in section 4 and lead on to section 5 with results on the benchmarks and analysis. Sections 6 and 7 are reflections on the project and detail how the project could be improved and what we learned. Finally, section 7 concludes the paper.


\section{Problem Characterization}
Start section here

\section{Dominant Approaches}
Start section here

\section{Methodology}%Need to reduce; the word count is currently 1010%
The planning for this project consisted of four phases that helped organize and streamline the research and development needed. These were researching, gathering data, preprocessing the data, and then a final analysis using the tools we defined. After the initial proposal, we first set out to research zoning types, residential zoning, and differences between the United States and other countries that have multi-purpose zoning laws. We narrowed down the countries to compare based on the accessibility of public datasets on zoning and standard of living. Besides the United States, these included English speaking countries such as England, Canada, Australia, etc. for the ease of translation, though we also considered other European nations as they also utilized mixed zoning development. With these considerations, we began to collect the necessary data.

The datasets needed to have some information about zoning, which we found was more common with cadastral datasets used for the purposes of taxing and recording land ownerships. As this is private information, their publicity depends on the laws of that country, which prevented us from accessing this for a lot of the countries we planned to compare to. In fact, we could not find any data on European or Asian countries that we thought would have comprehensible datasets. Even if they existed, gathering the data needed for quality of life would be another issue, which might not be compatible with the information for the US, so we planned to only focus on places within the United States. However, although the United States had county level datasets, they were not available as a collection. This meant that we could not use an API to download this data, and it forced us to navigate county websites, scavenging for any cadastral or parcel-based datasets. We were able to accumulate approximately 3 GB of data that spanned ~95 counties in the US, with most of them being in Ohio. This was because Ohio had the largest cadastral dataset available with the zoning information we required \cite{koordinates}. Others were selected from different regions of the US for a wider range of selection and distribution. These included Denver, Jefferson, and Weld County from Colorado, Dallas County and Houston from Texas, Placer County from California, and the city of Detroit, Michigan. In addition to zoning data, we also needed data to measure quality of living. We considered aspects such as unemployment, commute, poverty, mental health, occupancy, lot sizes, rent, etc. Some were easy to access using the US census, but others such as lot sizes and rent were more difficult to measure due to the differences in houses, condos, and apartments \cite{uscensus}. Our final data directory consisted of the parcel data of each place as well as files for quality of life, totaling 3.2 GB of data.

As all the data was collected separately for different use cases, it required preprocessing before the analysis. We decided to employ Apache Spark using Java and Gradle for both processes and the data itself was stored using HDFS. There are 6 Java files, and each consists of multiple spark jobs for preprocessing, calculating the zoning percentages, filtering out residential zoning, and processing quality of life data from the individual census datasets. The files are for the places described above categorized by their state or their name if they are singular in their state. This facilitated two types of logic in our code, one for working with JavaPairRDDs and another for JavaRDDs, but the structure for both is similar. As an example, Colorado.java reads three parcel files from HDFS for each of its counties. Each county name is set as the key and the extracted zoning information as the value. This allows us to use union to create a single RDD that encompasses all counties in Colorado. With this RDD, we can group by key and work with an iterable of zoning types. For example, we can use flatMapValues() to filter out all the values that are not residential using regex and count of all residential or just single-family lots for each county. Each job outputs its results locally to a directory called outputs that is organized by each program name. This is available to view in the project's repository and is what allows the final analysis using Python. The various other jobs correspond to calculating commute time, transportation types, poverty rates, housing occupancy, and other quality of life variables. They all follow a similar logic to one another and use RDDs to extract data from the corresponding datasets, match the data to the county, calculate percentages, and save that to the output directory. The main method coordinates all of these using a single spark context. The rest of the java files are structured in a similar way but there are changes based on how the zoning is defined, where it is located, and the type of RDD and thus the methods required. They also did not require a union of datasets as they were already completed. With this, we were able to use Spark to analyze the data and extract key information, reducing the size to less than 10 MB. The next step was to analyze the data, which we did using Python. We created a numpy ndarray based on all the outputs and used that for various analysis available to view in the notebook. This consisted of extracting features relating to quality of life and correlating them against the percentage of single-family zoning for every county. We chose to graph a linear function as well as a polynomial function to fit the data.

\section{Experimental Benchmarks}
Between our team, the cluster sizes were different due to issues with the setup, but it allowed us to benchmark the performance of both. One employed 2 live worker nodes while the other employed 10. This allowed us to measure the differences in performance and throughput in terms of worker nodes and size of the datasets. These results are shown in the table below and were measured with everything constant. Both clusters were run on the same day, time, and local machine consecutively. We chose not to run them in coordination as they use some of the same machines for the nodes, which could have affected the results due to resource availability. Instead, we alternated which cluster ran first and computed the averages of the results. The most noticeable results are in the Colorado row. In our benchmarks, this program took the longest to run despite it being a lot smaller than other datasets, which we assume is due to the union of three datasets. As mentioned above, this was the only program that utilized  unions, but everything else was similar. Besides this, there is a trend of the increase in time as the size of the dataset increases, which is expected as it requires more time to parse through a larger dataset. What is interesting is that for almost all programs, the cluster with 2 worker nodes outperformed the other except in 2 cases, Dallas and Detroit. We expected to observe the opposite effect, but we speculate this is due to the size of the datasets. The overhead of starting and stopping the nodes and distributing the work to 10 nodes detracted from the performance. If the data was larger, we think the larger cluster would have more impact, but it was not required for this use case.

\begin{table}[H]
  \centering
  \begin{tabular}{ |p{2cm}|p{2cm}|p{2cm}| p{2cm}| }
  \hline
    Dataset &Size &10 Nodes &2 Nodes\\
    \hline
      Colorado & 257.5 MB &2m 42s &2m 25s \\
      Dallas & 383.3 MB &5s &11s \\
      Detroit &129.9 MB &3s &8s \\
      Houston &721.9 MB &25s &20s \\
      Ohio &1.5 GB &24s &15s \\
      Placer &187.4 MB &10s &8s \\
    \hline
  \end{tabular}
  \end{table}

In our analysis of the data with Python, we graphed the disparity between single-family and multi-family zoning. It exhibits a large gap between the two as most of the places we examined had 90\% or above of its zoning allocated for single-family. In fact, looking at the means, single-family is ~89\% and multi-residential is ~10\%. We then used the percentage of single-family zoning in the county to correlate with each of the quality of life. Of those, the most significant were related to commute. Car usage seems to be positively correlated while the usage of public transport and walking seems to be negative. This aligns with one of our hypotheses that zoning affects commute types and times.

\section{Insights Gleaned}
Start section here

\section{Future}
Start section here

\section{Conclusions}
Start section here

\bibliographystyle{unsrt}
\bibliography{references}

\end{document}
\endinput